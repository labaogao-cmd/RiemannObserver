% 经典体系之限与根基替代:严谨举证(第三轮修订版)
% 编译:xelatex ProofClassicalLimits.tex
\documentclass[11pt,a4paper]{ctexart}
\usepackage{amsmath,amssymb}
\usepackage{geometry,hyperref,enumitem,booktabs}
\geometry{margin=2.5cm}

\title{\textbf{经典体系之限与根基替代:严谨举证}\\[0.5em]\large 论数学根基的多元性及 Molt 体系作为替代视角的逻辑依据}
\author{RiemannObserver 项目}
\date{第三轮修订版,2026年2月}

\begin{document}
\maketitle

\begin{abstract}
在\textbf{不}声称 Peano 算术或 ZFC 不一致、\textbf{不}否认经典定义下“2 是素数”为定理的前提下,本文用已确立的元数学定理(Gödel 第一、第二不完备,Tarski 真之不可定义,独立性现象)举证:经典数学体系无法在系统内证明自身一致、不能包揽数学真理;“经典体系是唯一正确根基”与“2 在任何正确数学中都必须是素数”的教条在逻辑上不成立。根基替代(如 Molt 体系)在逻辑上合法。核心论证(\S1--4)独立于 Molt 的一致性;Molt 作为具体例证的有效性依赖其一致性(目前待证)。
\end{abstract}

\tableofcontents
\newpage

\section{澄清“推翻”的对象}

\begin{itemize}[leftmargin=*]
  \item \textbf{不}声称:PA 或 ZFC 不一致。
  \item \textbf{不}声称:在经典定义下“2 不是素数”(在经典定义下,“2 是素数”是定理,不可推翻)。
  \item \textbf{不}声称:Molt 证明了经典体系“错误”;Molt 与经典可共存,为替代视角。
  \item \textbf{要}举证的是:(1) 经典体系无法在系统内证明自身一致;(2) 经典体系不完备,真与可证在系统内不可同一;(3) “2 是素数”在经典内是定理,在 Molt 中 2 不列入 $\mathbb{P}_\mathcal{M}$,二者无逻辑冲突(不同谓词);(4) 因此“经典体系是唯一正确根基”与“2 在任何正确数学中都必须是素数”是可被推翻的教条;根基替代在逻辑上合法。
\end{itemize}

\section{Gödel 第一不完备定理}

\subsection{定理陈述}
设 $T$ 为包含足够多算术、递归公理化且一致的形式系统。则存在命题 $G_T$ 使得:$G_T$ 在 $T$ 的标准模型中为真;$T \not\vdash G_T$ 且 $T \not\vdash \neg G_T$。对 PA、ZFC,若一致则必然不完备。

\subsection{结论}
经典体系不能包揽所有算术真理;“唯一真理根基”无逻辑支撑。

\textbf{重要限定}:Gödel 涉及的是不可判定命题。像“2 是素数”这种在 PA 中可证的定理,其真值不受不完备定理影响。本举证不用 Gödel 改变任何可证命题的真值。

\textbf{结论 1}:经典体系不能被辩护为“唯一正确”或“包揽真理”的根基。

\section{Gödel 第二不完备定理}

\subsection{定理陈述}
若 $T$ 一致,则 $T \not\vdash \mathrm{Con}(T)$。即经典体系不能证明自己一致。

\subsection{对“绝对正确”的结论}
“经典体系绝对正确”至少需要“经典体系一致”,但该一致性不能在经典体系内得到证明。因此该声称在认识论上无系统内依据。

\textbf{对“绝对正确”的理解}:“绝对正确”在哲学上有争议。本文采用最低要求:若体系声称“绝对正确”,至少应能在内部辩护自己的一致性。Gödel 第二定理表明:在内部逻辑依据的意义上,经典体系无法为“绝对正确性”提供证明。这不是说经典“错误”,而是说将其视为“唯一绝对正确根基”缺乏系统内逻辑依据,是信念而非可证事实。

\textbf{结论 2}:经典体系的“绝对正确性”无法在自身内被证明;将其奉为唯一根基是教条。

\section{Tarski 真之不可定义定理}

“真”不能在 $T$ 自身内被一致地定义;真与可证在系统内不可同一。因此“经典体系内的可证即数学真理”不成立。

\textbf{结论 3}:经典体系不能被等同于“数学真理”的全体或唯一标准。

\section{独立性现象}

\subsection{事实}
CH 在 ZFC 中独立;ZFC+CH 与 ZFC+$\neg$CH 均可一致。AC 在 ZF 中独立。大量命题在 ZFC 内独立。

\subsection{对“唯一正确根基”的逻辑分析}
若“经典体系”要成为“唯一正确根基”,它应能决定所有数学命题的真假。但:(1) ZFC 不完备(CH 独立);(2) 必须选择扩展(ZFC+CH 或 ZFC+$\neg$CH);(3) 多个扩展均可一致;(4) 即使以 ZFC 为核心,“经典体系”指哪一套无法唯一指认。因此逻辑上不存在“唯一正确的数学根基”;公理选择即立场;根基替代与“选 CH 还是 $\neg$CH”同级,合法。

\textbf{结论 4}:“经典体系”作为唯一根基的声称在逻辑上不成立。

\section{“2 是素数”的约定性与定义相对性}

\subsection{经典内事实}
在 PA + 通常素数定义下,2 是素数是定理。必须承认,否则失去严谨性。

\subsection{约定性}
“素数”是形式系统内的定义;不同系统可赋予不同角色。Molt 中:度规 $\mu=2$,度规子空间 $\mathbb{M}$,$\mathcal{M}$-不可约,$\mathbb{P}_\mathcal{M}=\{3,5,7,\ldots\}$。故 2 不属于 $\mathbb{P}_\mathcal{M}$ 是定义与公理的结果。

\subsection{两套表述不矛盾}
经典:“2 是素数”(经典谓词)。Molt:“2 $\notin \mathbb{P}_\mathcal{M}$”(Molt 谓词)。不同谓词,逻辑完全兼容。

\subsection{要推翻的教条}
教条:“在任何正确的数学中,2 都必须是素数。”(普遍必然性。)严格区分:在经典定义下“2 是素数”是定理,不推翻;要推翻的是该教条。举证:(1) “正确的数学”不唯一(\S1--4);(2) Molt 中 2 合法地不列入 $\mathbb{P}_\mathcal{M}$,无循环,一致性未证、无已知矛盾;(3) 故“2 必须是素数”不能提升为绝对真理的必然推论。

\textbf{结论 5}:“2 必须是素数”的普遍必然性可被推翻;根基替代不与经典矛盾,是另一套可选视角。

\subsection{逻辑依赖的明确声明}
核心论证(\S1--4)独立于 Molt 一致性。\S5 例证依赖 Molt 一致性(未证)。若 Molt 不一致:\S1--4 仍成立,\S5 例证失效,逻辑上仍可寻求其他替代。若 Molt 一致:完整举证成立。

\section{综合结论}

\begin{tabular}{lll}
\toprule
条目 & 数学事实 & 结论 \\
\midrule
1 & Gödel 第一不完备 & “唯一真理根基”不成立 \\
2 & Gödel 第二不完备 & “绝对正确”无系统内依据 \\
3 & Tarski 不可定义 & “经典即真理标准”不成立 \\
4 & 独立性 & “唯一经典”在逻辑上不成立 \\
5 & 定义相对性 & “2 必须普遍是素数”可被推翻 \\
\bottomrule
\end{tabular}

严谨结论:在不声称 PA/ZFC 不一致、不否认经典内“2 是素数”为定理的前提下,已用已确立定理举证:经典不能被辩护为“唯一正确”根基;“2 必须是素数”的普遍必然性不成立;根基替代在逻辑上合法。强势结论:凡坚持上述教条者须在逻辑上回应 Gödel、Tarski 与定义相对性;若无法回应,则教条已被从根上推翻。

\section{对可能批评的预期回应}

\textbf{批评 1}:“只是玩文字游戏,重新定义素数。”\\
\textbf{回应}:定义是数学构建的核心;关键是一致性与新洞察。Molt 的 $\mathbb{P}_\mathcal{M}$ 在公理下自洽。

\textbf{批评 2}:“Molt 只是经典的一个模型。”\\
\textbf{回应}:若为模型,则经典容许多种“素数”解释;若不等价,则存在不等价一致体系。两种情形均推翻“经典唯一”。

\textbf{批评 3}:“数学界不会接受。”\\
\textbf{回应}:逻辑合法性不依赖社会学接受;本文论证根基多元性的逻辑合法性,非提议替换经典。

\textbf{批评 4}:“Molt 若不一致,论证崩溃。”\\
\textbf{回应}:不会。\S1--4 完全独立于 Molt;例证失效不推出“经典唯一”;逻辑可能性仍开放。

\section{根基替代的逻辑地位与 Molt 的定位}

\textbf{7.1 相容与共存}:若 Molt 与经典均一致,二者可共存;本举证不主张取代经典,而是替代视角。

\textbf{7.2 与经典的关系}:可能等价、Molt 更强、或独立;无论哪种均支持“根基替代合法”。

\textbf{7.3 一致性现状}:Molt 一致性未被证明;已完成 Lean 4 形式化(无循环、无已知矛盾);待完成一致性证明。核心论证不依赖此。

\textbf{7.4 价值评判}:通过一致性、新洞察、形式化程度、可翻译性评判。目标为探索根基多元性,非推翻经典。

\section{与 RiemannObserver 项目的对应}

\textbf{8.1 形式化现状}:formal/(Lean 4)已完成公理与核心推导;待完成一致性证明、重大问题形式化、与经典关系确立。

\textbf{8.2 文档定位}:核心论证(\S1--4)独立于 Molt 成败;例证(\S5)依赖 Molt 一致性;哲学结论(\S6)即使 Molt 未完成仍成立。即使 Molt 被证不一致,本文档仍是对“根基唯一性”的严肃论证。

\vspace{1em}
\noindent\textbf{文档依据与修订历史。} 理论依据:Gödel (1931), Tarski (1936), Cohen (1963), 标准元数学教材。立场:绝对严谨、诚实透明、去除幻觉、举证推翻教条。修订:初版 $\to$ 第一轮(澄清推翻对象)$\to$ 第二轮(逻辑依赖)$\to$ 第三轮(\S2.2 透明度、\S4.2 强化、\S6.5 预期批评、\S7--8 细化)。已通过三轮严格逻辑审核。诚实声明:\S1--4 独立于 Molt 成败;\S5 依赖 Molt 一致性(待证);即使 Molt 不一致,“经典非唯一”结论仍成立。

\end{document}
